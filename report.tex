\documentclass[twocolumn]{ctexart}
\usepackage{lipsum,mwe,cuted}
\usepackage{float}
\usepackage{caption}
\usepackage{fancyhdr}
\usepackage{amsthm}
\pagestyle{fancy}
\usepackage{listings}
\usepackage{enumerate}

\usepackage{geometry}
\geometry{a4paper,scale=0.8}
\title{\textbf{Manderbrot Set 的生成和探索}}
\author{ 邵盛栋  \\\ 信息与计算科学 3200103951 }
\date{\vspace{-2em}}

\stripsep 8pt
\pagestyle{plain}
\newtheorem{thm}{定理}
\bibliographystyle{unsrt}
\ctexset{section={format={\Large\bfseries}}}
\newcommand{\upcite}[1]{\textsuperscript{\textsuperscript{\cite{#1}}}}


\begin{document}
	
	\maketitle
	
	\begin{strip}
		
		\noindent  \textbf{摘要} \quad 对于Mandelbrot集,它展示了一个精细且无限复杂的边界,随着放大倍数的增加,它会显示出越来越精细的递归细节,本文将在研究Mandelbrot集的基础上,首先探究Mandelbrot集的理论知识,寻找适当的算法并利用C++语言编写程序,最终实现Mandelbrot集可视化。
		\\
	\end{strip}
	
	
	\section*{1  引言}
	Mandelbrot集是复平面中c值的集合,其临界点的轨道为$z=0$在二次映射的迭代下
	\[z_{n+1}=z_{n}^{2}+c\]
	仍然有界。因此,复数c是Mandelbrot集中的元素,不妨假设开始时$z_{0}=0$,在一次次迭代后$|z_{n}|$对于任意$n>0$有界,则对应的$c$即属于Mandelbrot集。
	
	例如,对于$c=1$,序列是$ 0,1,2,5,26,\dots $,趋于无穷,则说明1不是Mandelbrot集中的元素;另一方面,对于$ c=-1 $,序列是$ 0,-1,0,-1,0,\dots $,是有界的,则说明-1是Mandelbrot集中的元素\cite{devaney2006unveiling}。
	\section*{2 问题背景}
	Mandelbrot集起源于复动力学,这是20世纪初法国数学Pierre Fatou和Gaston Julia首次研究的领域。1978年,罗伯特·W·布鲁克斯和彼得·马特尔斯基首次定义并绘制了这种分形,作为Kleinian groups研究的一部分。1980年3月1日,Benoit Mandelbrot在纽约约克镇高地的IBM托马斯J.沃森研究中心首次看到了该集合的可视化。
	
	Mandelbrot在1980年发表的一篇文章中研究了二次多项式的参数空间。Mandelbrot集的数学研究真正始于数学家Adrien Douady和John H. Hubbard (1985)的工作,他们建立了许多其基本属性,并命名该集合以纪念Mandelbrot在分形几何方面的有影响的工作。
	
	1985年8月的《科学美国人》的封面文章向广大读者介绍了计算Mandelbrot集的算法。封面由不来梅大学的Peitgen 、Richter 和Saupe创作。Mandelbrot集在1980年代中期作为计算机图形演示变得突出,当时个人计算机变得足够强大,可以以高分辨率绘制和显示该集。
	\section*{3 数学理论}
	Mandelbrot集是一个紧集,由于它是闭合的并且包含在以原点为圆心的半径为2的闭合圆中,即点c属于Mandelbrot集当且仅当$ |z_{n}|\leq2 $对于所有$ n\geq0 $有界\cite{douady1986julia},则有如下定理:
	\begin{thm} 
	$ 0 $的轨道趋向于无穷大当且仅当在某个点它的模$ \textgreater2 $。
	\end{thm}
	
	该定理可分为以下两个定理:
	\begin{thm} 
		若$ c\in M $,则$ |c|\leq 2 $。
	\end{thm}
	\begin{thm} 
		若$ c\in M $,则$ |z_{n}|\leq 2,(n=1,2,\dots) $。
	\end{thm}
	下面是对定理三的证明:
	\begin{proof}
		要证明若$ |z_{n}|>2,(n=1,2,\dots) $,则$ c\notin M $\\
		首先分别讨论$ |c|>2 $与$ |c|\leq 2 $两种情形\\
		由定理2,$ z_{n}>2,(n=1,2,\dots) $且$ |c|>2 $时,$ c\notin M $.\\
		接着要证明$ |c|\leq 2 $时的情况:\\
		假设$ |z_{n}|>2 $,因为$ |c|\leq 2 $,所以$ |z_{n}|>|c| $,而\\
		\[|z_{n+1}|=|z_{n}^{2}+c|\geq|z_{n}^{2}|-|c|=|z_{n}|^{2}-|c|\]
		因为$ |z_{n}|>|c| $\\
		\[|z_{n}|^{2}-|c|>|z_{n}|^{2}-|z_{n}|\]
		由$ |z_{n}|>2 $,左右同乘$ |z_{n}| $再减去$ |z_{n}| $可得到下式\\
		\[|z_{n}|^{2}-|z_{n}|>2|z_{n}|-|z_{n}|=|z_{n}|\]
		由以上可知$ |z_{n+1}|>|z_{n}| $\\
		由数学归纳法可知$ 2<|z_{n}|<|z_{n+1}|<|z_{n+2}|<\dots $,可以看出随着迭代次数的增加$ |z_{n}| $递增并且发散。\\
		所以在$ |z_{n}|>2,(n=1,2,\dots) $且$ |c|\leq 2 $的情况下也是则$ c\notin M $。\\
		综上所述,若存在$ |z_{n}|>2 $,则$ c\notin M $
	\end{proof}
	该定理只适用于$ z\mapsto z^{2}+c $,但通过修改边界值2,可以适用于其他多项式族。这里的边界值不取决于c,但在其他多项式族中可能取决于c\cite{weisstein2002mandelbrot}。

	\section*{4 算法}
	算法伪代码如下:
	\begin{verbatim}
	Choose a maximal iteration number N
	For each pixel p of the image:
	  Let c be the complex number represented 
	  by p
	  Let z be a complex variable
	  Set z to 0
	  Do the following N times:    
		  	  If |z|>2 then color the pixel white, 
		  	  end this loop prematurely, go to the
		  	  next pixel Otherwise replace z by 
	     z*z+c
	  If the loop above reached its natural
	  end: 
	  color the pixel p in black
	  Go to the next pixel
	\end{verbatim}
	利用C++实现算法的具体流程为:
	\begin{enumerate}[(1)] 
		\item 创建bmp图像文件(通过RGB三种颜色显现的图像),图像中的像素可作为坐标轴当中的点,并给定原点坐标;
		\item 定义一种迭代形式,来检验对应的点c是发散还是收敛;
		\item 检验图像中的每一个点(像素),并对收敛和发散的点进行不同的着色处理,绘制形成目标图像。
	\end{enumerate}
	
	\section*{5 数值算例}
	设置ox,oy,dimension与迭代次数N分别为0,0,4,100,在run文件中输入如下命令
	\begin{verbatim}
		./test picture1.bmp 0 0 4 100
	\end{verbatim}
	可得到picture1,如图\ref{fig:1}所示。
	
	我们可以调节N的大小来得到更加精细的Mandelbrot集的图案,不妨设置N的值为200,即在run文件中输入命令:
	\begin{verbatim}
		./test picture2.bmp 0 0 4 200
	\end{verbatim}
	可得到picture2,如图\ref{fig:2}所示。
	
	我们还可以通过调节ox,oy,dimension的值来观察Mandelbrot集的细节部分,在run中输入如下命令:
	\begin{verbatim}
		./test picture3.bmp 0 0.5 1 200
	\end{verbatim}
	从而得到想要观察的位置的放大图,如图\ref{fig:3}所示。
	
	最后,我们还可以改变图案的颜色,使其更加美观,可视化效果更好,如图\ref{fig:4}所示。
    \begin{figure}[htbp]
      	\includegraphics[width=\linewidth]{picture1.bmp}
      	      	\caption{picture1}
      	\label{fig:1}
    \end{figure}
 	\begin{figure}[htbp]
 		\includegraphics[width=\linewidth]{picture2.bmp}
 		\caption{picture2}
 		\label{fig:2}
 	\end{figure}
 	\begin{figure}[htbp]
 		\includegraphics[width=\linewidth]{picture3.bmp}
 		\caption{picture3}
 		\label{fig:3}
 	\end{figure}
 	\begin{figure}[htbp]
 		\includegraphics[width=\linewidth]{picture4.bmp}
 		\caption{picture4}
 		\label{fig:4}
 	\end{figure}
 
	\section*{6 结论}
	Mandelbrot集的迭代次数可间接反映发散程度,本文中的代码将收敛与发散两种不同情况进行区分,并赋予黑色与白色两种颜色的值,由此得到可视化效果效果。在此基础上,我们仍可以通过改变维数以及迭代次数,得到一些新的研究思路与方法。
	
	\bibliographystyle{unsrt}
	\bibliography{literature}
\end{document}
